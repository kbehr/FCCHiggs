\section{Detector influences}

To investigate the detector influences on the two channels, we do a comparison between the existing CMS detector and the proposed detector for CLIC, similar to the ILD detector. Cardfiles for both are provided with the \textsc{heppy} framework.
To investigate the effects of the different subdetectors, each subdetector is placed inside CMS. Radii are adapted so that there is no overlap or empty space bigger than in CMS between the subsystems, but their thickness is preserved. The magnetic field is reduced to 2\,T since this is the likely magnetic field of a detector at FCC-ee.

\subsection{Statistical evaluation}
The main quantity of interest is the signal yield and its uncertainty. To obtain the latter, Poissonian smearing of the histograms is used. Results are verified by propagating the errors of the parameters of the fitted function to its integral.

\subsection{Results}
% placeholder for moneyplot
% fill in correct numbers
In the four-jet channel, the subdetector showing the greatest improvement is the hadronic calorimeter (HCAL). Signal yield is increased by a third, the attainable precision of the $\sigma_\mathrm{hZ} \times \mathrm{BR(h \rightarrow b\bar{b})}$ is reduced to 1.49\,\% compared to 2.00\,\% with the CMS detector. The tracker offers smaller improvements, boosting the precison of the $\sigma \times $BR measurement to 1.71\,\%