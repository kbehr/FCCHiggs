\section{Event generation and selection}
All events are generated with \textsc{Pythia~8.2} \cite{pythia82}. The expected beam energy spread at the FCC-ee (0.17\,\% on single beam, 0.12\,\% on center-of-mass energy) is considered by randomly variating $\sqrt{s}$ during the rescaling done in each analysis. Jet clustering is done using \textsc{FastJet} \cite{fastjet}.
The event selection follows \cite{lep3-note} and is implemented with the \textsc{heppy} framework \cite{heppy}, which contains the fast detector simulation \textsc{papas}.

\subsection{Four-jet channel}
To exclude events from the Z$^*/\gamma^* \rightarrow \mathrm{q\bar{q}}$ background, jets are formed with the anti-$k_T$ algorithm \cite{antikt_algo} with a minimum transverse momentum of 8\,GeV. If an event has at least four jets, the reconstructed particles are clustered into four jets using the $k_T$ algorithm \cite{kt_algo} in the exclusive mode. This is done because the further analysis requires four jets. Combining jets by hand was found  to be very prone to errors.
Every jet is required to have at least five reconstructed particles,  at least one of which has to be a charged hadron. This rejects Z decays into pairs of $e$, $\mu$ or $\tau$. Decays into neutrinos are rejected by requiring the visible mass\footnote{The mass of an object whose four-momentum is the sum of the four momenta of all jets.} of an event to be bigger than 180\,GeV. Now the jet energies are rescaled so that the sum of their energies equals the center-of-mass energy of the event while keeping the directions fixed. 
All rescaled energies have to be positive.
To exclude fully hadronic ZZ or WW decays, both $\Delta_\mathrm{ZZ}$ and $\Delta_\mathrm{WW}$, defined as
\begin{equation}
\Delta_\mathrm{ZZ, WW} = \min_{i, j, k, l} [(m_{ij} - m_\mathrm{W, Z})^2 + (m_{kl} - m_\mathrm{W, Z})^2],
\end{equation}
have to be larger than 10\,GeV. Here, $i, j, k, l$ are the four jet indices. Throughout this thesis, $m_\mathrm{h} = 125\,$\,GeV, $m_\mathrm{W} = 80$\,GeV and  $m_\mathrm{Z} = 91$\,GeV have been used as theoretical values.

A candidate Higgs boson is a jet pair with a mass $m_{12} > 100$\,GeV, whereas $m_{34}$ has to be between 80 and 110\,GeV. Both jets forming $m_{12}$ have to be b-tagged. If several combinations satisfy this condition, the one with the Higgs boson mass closest to 125\,GeV is taken. The Higgs boson mass is calculated as
\begin{equation}
  m_\mathrm{h} = m_{12} + m_{34} - m_\mathrm{Z}.
\label{eq:higgsmass}
\end{equation}
This quantity has a slightly better resolution than $m_{12}$.
The signal is fitted to the sum of a Breit-Wigner and a Gaussian function, the background to a third order polynomial.