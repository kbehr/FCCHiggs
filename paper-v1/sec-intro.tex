The introduction goes here.

\section{Introduction}
While the LHC has found no evidence so far for physics Beyond the Standard Model (BSM), there are plenty of phenomena that cannot be explained by the standard model. Astronomical observations have given evidence for the existance of Dark Matter: Stable particles or objects (if they weren't, they would have decayed by now) that hardly interact with electromagnetic radiation (hence \emph{Dark} Matter) \cite{pdg14}. 
Other open questions are the baryon asymmetry of the universe and the hierarchy problem of the Standard Model (SM).

A new generation of particle colliders, such as the Future Circular Collider (FCC), could investigate some of these open questions. The FCC is an ongoing study for a proton-proton, $e^+ e^-$ and an $e$-proton collider hosted in a 100\,km tunnel at CERN.

We investigate the detector influence on the measurement of Higgs boson properties at the FCC-ee, the $e^+ e^-$ variant of the project.
At the FCC-ee, Higgs bosons could be produced in large quantities in association with Z bosons via the \emph{Higgsstrahlung} process $e^+ e^- \rightarrow \mathrm{Zh}$. Operation at $\sqrt{s} = 240$\,GeV is expected to be the best compromise between cross section and operation costs of the collider for this process.

Since electrons are -- unlike protons -- elementary particles, the energy and momentum of the collision is known and can be used as a constraint in the analysis. This enables for example the measurement of invisible decays of the Higgs boson by observing the mass recoiling against a Z boson decaying in visible particles.